\documentclass[11pt]{article}
\usepackage[utf8]{inputenc}
\usepackage{amsfonts}
\usepackage{natbib}
\usepackage{graphicx}
\usepackage{amsmath}
\usepackage{amssymb}
\usepackage{mathrsfs} % Cursive font
\usepackage{graphicx}
\usepackage{ragged2e}
\usepackage{fancyhdr}
\usepackage{nameref}
\usepackage{wrapfig}
\usepackage{hyperref}

% for C
\usepackage{xcolor}
\usepackage{listings}

\definecolor{mGreen}{rgb}{0,0.6,0}
\definecolor{mGray}{rgb}{0.5,0.5,0.5}
\definecolor{mPurple}{rgb}{0.58,0,0.82}
\definecolor{backgroundColour}{rgb}{0.95,0.95,0.92}

\lstdefinestyle{CStyle}{
    backgroundcolor=\color{backgroundColour},   
    commentstyle=\color{mGreen},
    keywordstyle=\color{magenta},
    numberstyle=\tiny\color{mGray},
    stringstyle=\color{mPurple},
    basicstyle=\footnotesize,
    breakatwhitespace=false,         
    breaklines=true,                 
    captionpos=b,                    
    keepspaces=true,                 
    numbers=left,                    
    numbersep=5pt,                  
    showspaces=false,                
    showstringspaces=false,
    showtabs=false,                  
    tabsize=2,
    language=C
}


\usepackage{mathtools}
\usepackage{xparse} \DeclarePairedDelimiterX{\Iintv}[1]{\llbracket}{\rrbracket}{\iintvargs{#1}}
\NewDocumentCommand{\iintvargs}{>{\SplitArgument{1}{,}}m}
{\iintvargsaux#1}
\NewDocumentCommand{\iintvargsaux}{mm} {#1\mkern1.5mu,\mkern1.5mu#2}

\makeatletter
\newcommand*{\currentname}{\@currentlabelname}
\makeatother

\usepackage[a4paper,hmargin=1in, vmargin=1.4in,footskip=0.25in]{geometry}

\graphicspath{ {./images/} }


%\addtolength{\hoffset}{-1cm}
%\addtolength{\hoffset}{-2.5cm}
%\addtolength{\voffset}{-2.5cm}
\addtolength{\textwidth}{0.2cm}
%\addtolength{\textheight}{2cm}
\setlength{\parskip}{8pt}
\setlength{\parindent}{0.5cm}
\linespread{1.5}

\pagestyle{fancy}
\fancyhf{}
\rhead{Trabajo Práctico Final - Sullivan}
\lhead{Estructuras de Datos y Algoritmos I}
\rfoot{\vspace{1cm} \thepage}

\renewcommand*\contentsname{\LARGE Índice}

\begin{document}

\begin{titlepage}
    \vspace{-1cm}
    \hspace{-1.2cm}\includegraphics[scale= 0.8]{header2.png}
    \begin{center}
        \vfill
        \vfill
            \vspace{0.7cm}
            \noindent\textbf{\Huge Trabajo Práctico Final}\par
            \vspace{.5cm}
        \vfill
        \noindent \textbf{\huge Alumna:}\par
        \vspace{.5cm}
        \noindent \textbf{\Large Sullivan, Katherine}\par
 
        \vfill
        \large Universidad Nacional de Rosario \par
        \noindent\large 2021
    \end{center}
\end{titlepage}

A continuaci\'on se presenta un informe detallando las decisiones en general y particularidades en el dise\~{n}o y desarrollo 
del Trabajo Pr\'actico Final de la asignatura Estructuras de Datos y Algoritmos I. 

\section{M\'odulos del programa}

\section{Estructuras de datos utilizadas}

\subsection{Tabla Hash}
Agenda viene a representar
Lookup mas mportante en una genda

\begin{lstlisting}[style = CStyle]
typedef struct {
  CasillaHash *tabla;
  unsigned numElems;
  unsigned capacidad;
  FuncionHash hash;
  FuncionHash hash2;
  Arbol arbol_nombre;
  Arbol arbol_apellido;
  Arbol arbol_edad;
  Arbol arbol_tel;
} TablaHash;
\end{lstlisting}

\subsection{Árboles AVL}

\begin{lstlisting}[style = CStyle]
typedef struct _Nodo {
  void *dato;
  int idx;
  struct _Nodo *izq;
  struct _Nodo *der;
  int alt;
} Nodo;

typedef Nodo *Arbol;
\end{lstlisting}

\begin{lstlisting}[style = CStyle]
typedef struct _STNodo {
  int idx;
  struct _STNodo *izq;
  struct _STNodo *der;
  int alt;
} STNodo;

typedef STNodo *STree;
\end{lstlisting}

\subsection{Listas simplemente enlazadas}

\begin{lstlisting}[style = CStyle]
typedef struct _SNodo {
  int dato;
  struct _SNodo *sig;
  int cant;
} SNodo;

typedef SNodo *SList;
\end{lstlisting}

\subsection{Listas de acciones}

\begin{lstlisting}[style = CStyle]
typedef struct {
  int tipo;                     // 1 agregar - 2 eliminar - 3 editar 
  char *nombre;
  char *apellido;
  char **tel;
  int *edad;
} Accion;
\end{lstlisting}

Mencionar lo de nodo

\begin{lstlisting}[style = CStyle]
typedef struct {
  AccNodo *head;
  AccNodo *tail;
  int elems;
  int cap;
} AccList;
\end{lstlisting}

\subsection{Otras estructuras}

\section{Algoritmos de inter\'es}

    \subsection{Deshacer/Rehacer}

    \subsection{Guardar ordenado}

    \subsection{Buscar por suma de edades}

\section{Dificultades encontradas}

    \subsection{Sobre la impresi\'on en la b\'usqueda de suma de edades}

    \subsection{Sobre la duplicaci\'on de la informaci\'on}

\section{Decisiones particulares}

    \subsection{Sobre cu\'ando reinicializar la lista de acciones de rehacer}

    \subsection{Sobre pisar la informaci\'on de un contacto}

    \subsection{Sobre resizing din\'amico de la agenda}
    Costoso pero mejor experiencia de usuario. No real. 

\section{Compilaci\'on e invocaci\'on}
Para la compilaci\'on del programa la entrega cuenta con un archivo Makefile. Para producir el archivo ejecutable basta 
con correr alguno de los siguientes comandos: 

\begin{itemize}
    \item \verb|make|: adem\'as de generar un ejecutable para el uso del programa borra todos los archivos objeto de la carpeta actual
    \item \verb|make all|: \'idem make
    \item \verb|make main|: solo produce el ejecutable main
\end{itemize}

\section{Bibliograf\'ia}

\end{document}